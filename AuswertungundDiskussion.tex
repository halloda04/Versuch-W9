\section{Auswertung}

\subsection{Seebeck-Koeffizient des Thermoelements}

Um den Seebeck-Koeffizient des Thermoelements zu bestimmen kann die Gleichung für zwei Materialien A und B und bei einem Temperaturgefälle von $T_2 - T_1$  aus Gleichung (\ref{}) hergeleitet werden. 
\begin{equation}
  U_{th} = \int_{T_1}^{T_2} \alpha_A~ \text{d}T - \int_{T_1}^{T_2} \alpha_B~ \text{d}T = \int_{T_1}^{T_2} \alpha_{AB}~ \text{d}T
\end{equation}
Mit $\alpha_{AB} = \alpha_A - \alpha_B$ vereinfacht sich das zu

\begin{equation}
U_{th} = \alpha_{AB} \cdot (T_2 - T_1)
\label{gl:uth}
\end{equation}

Trägt man die Temperaturdifferenz, welche in diesem Fall einfach die gemessene Temperatur in Grad Celsius ist, da die Referenz Temperatur null Grad Celsius betrug, gegen die gemessene Spannung auf, so ergibt sich Diagramm \ref{bi:Ut1}.

\begin{figure}[H]
    \centering
    \includegraphics[width=\linewidth, keepaspectratio]{Bilder/Ut1.png}
    \caption{Die Thermopannung gemessen in mV aufgetragen gegen die Temperaturdifferenz. Die Ausgleichgerade wurde mittels Numpy.polyfit ersten Grades erstellt. Die Grenzgeraden wurde Mittels einer Funktion welche die Grenzeraden so legt, dass $\frac{2}{3}$ aller Punkte zwischen den beiden Grenzgeraden liegen, erstellt. Alle drei Geraden laufen durch den Schwerpunkt.}
     \label{bi:Ut1}
\end{figure}

Mit der Steigung der Ausgleichsgerade aus Tabelle \ref{tb:steig1} ergibt sich für das gegebene Thermoelement  ein Seebeck-Koeffizient von 
\begin{equation}
  \alpha_{\mathrm{AB}} = 4,49 \cdot 10^{-5} \frac{\mathrm{V}}{\mathrm{K}}
\end{equation}

Der Fehler des Seebeck-Koeffizient ergibt sich aus den Steigungen der Grenzgeraden welche in Tabelle \ref{tb:steig1} abgebildet sind.

\begin{table}[H]
    \centering
    \begin{tabular}{c c}
        \hline
        Gerade &  Steigung ($\frac{\mathrm{mV}}{\mathrm{K}}$)\\
        \hline
        Ausgleichsgerade   & 0,0449    \\
        Grenzgerade 1   & 0,0426   \\
        Grenzgerade 2 & 0,0472   \\
    
    \end{tabular}
    \caption{Die Steigungen der Ausgleichs- und Grenzgeraden}
    \label{tb:steig1}
\end{table}

Verwendet man diese Werte ergibt sich folgender Fehler.

\begin{equation}
  \Delta \alpha_{\mathrm{AB}} = \frac{0,0472 \frac{\mathrm{mV}}{\mathrm{K}} - 0,0426 \frac{\mathrm{mV}}{\mathrm{K}}}{2} = 0,0023 \frac{\mathrm{mV}}{\mathrm{K}} 
  \label{seefehler}
\end{equation}

Dass Ergebniss ist dann folgendes.

\begin{equation}
  \alpha_{\mathrm{AB}} = (4,49 \pm 0,23) \cdot 10^{-5} \frac{\mathrm{V}}{\mathrm{K}}
\end{equation}

Der relativer Fehler $\frac{\Delta \alpha}{\alpha}$ beträgt ca. 5\%. Dieser Fehler scheint realistisch, da das Spannungsmessgerät als auch das Temperaturmessgerät Messungenaugkeiten haben. Des weiteren wurde beim erhitzen des Wassers nicht umgerührt, was dazuführen kann das bei den Kontaktpunkten des Thermoelements eine andere Temperatur als bei dem Thermometers herscht, was zu einer Verfälschung des Ergebnisses führen könnte. Dieser Fehler ist dann aber höchstwahrscheinlich konstant, also nicht in $\Delta \alpha$ erkennbar, da sich das Wasser trotzdem konstant erwärmt. Es wär somit ein Systematischer Fehler. Zudem ist vermerkt das der Thermometer während der Messung verrutscht ist. Da dies aber, zumindest nicht deutlich, in Abbildung \ref{bi:Ut1} zu erkennen ist, ist davon auszugehen das der Gerade beschriebene Effekt nicht gravierend ist.



Wie vermutet ist ein linearer Zusammenhang zwischen der Temperaturdifferenz und der Spannung, wie von Gleichung \ref{gl:uth} vorhergesagt, in Abbildung \ref{bi:Ut1} zu sehen.
% Vergleicht man diesen Wert mit dem theoretischen wert von Gleichung \ref{} ergibt sich folgender Seebeck-Koeffizient. 

% \begin{equation}
% \alpha = \frac{3k_\mathrm{B}}{2e} = 12,9 \cdot 10^{-5} \frac{\mathrm{V}}{\mathrm{K}
% \end{equation}

\subsection{Peltier Element}
Stellt man den Zeitlichen Verlauf der Thermospannung $U_{\mathrm{th}}(t)$ für die verschiedenen Stromstärken grafisch da, so erhält man Abbildung \ref{bi:Utversc}


\begin{figure}[H]
    \centering
    \includegraphics[width=\linewidth, keepaspectratio]{Bilder/utverschiedenespannugen.png}
    
    \caption{Die Spannung gemessen in mV aufgetragen gegen vergangene Zeit t in s nach dem Einschalten der Stromquelle bzw. nach dem ändern der Stromstärke.}
    \label{bi:Utversc}
\end{figure}

Der Sprung der Kurve von $I = 2,4 $ A bei ca $t = 80$s  lässt sich durch das Spannungsmessgerät erklären. Da dieses keine Spannungen unter $\left|0,5\right|$ mV Messen kann, zeigt dieses bis zum Erreichen von $U_{\mathrm{th}} = -0,5$mV eine Spannung von  $U_{\mathrm{th}} = 0$mV  an und springt dann plötzlich auf von $U_{\mathrm{th}} = -0,5$ mV. Die beste Effektivkühlung der gemessenen Werte, erhält man bei einer Stromstärke von $I = 3,2 $ A. Dies liegt daran das bei einer größeren Stromstärke die Erwärmung durch den ohmschen Wiederstand stärker wächst als die Abkühlung durch das Peltier-Element. 

Da das selbe Thermoelement wie bei der ersten Messung verwendet wurde und die Referenzthemperatur ebenfalls 0°C ist, kann $T_{\mathrm{O}}$ also die Temperatur am Oberseite des Peltier Element mit der gemessenen Thermopannung $U_{\mathrm{th}}$ errechnet Werden. Die Formel dafür

\begin{equation}
T_{\mathrm{O}} = \frac{U_{\mathrm{th}}}{\alpha_{\mathrm{AB}}}
\end{equation}

folgt aus Gleichung \ref{Thermospannung} Der Größtfehler errechnet sich mit
\begin{equation}
\label{eq:Groesztfehler}
\Delta T_{\mathrm{O}} = 
\pm \left(
\left| \frac{\partial T_O}{\partial U_{\mathrm{th}}} \right| \, \Delta U_{\mathrm{th}} 
+ \left| \frac{\partial T_O}{\partial \alpha_{AB}} \right| \, \Delta \alpha_{AB} 
\right)
= \pm \left(
\frac{\Delta U_{\mathrm{th}}}{\alpha_{AB}} + \frac{U_{\mathrm{th}} \, \Delta \alpha_{AB}}{\alpha_{AB}^2} 
\right) .
\end{equation}

Dabei FHELR

In Tablle \ref{tb:to} sind die daraus folgenden Temperaturdifferenzen zu sehen. 

\begin{table}[H]
    \centering
    \begin{tabular}{c | c | c}
        \hline
        $ I $(A) &  $T_{\mathrm{O}}$ (°C) & $\Delta T_{\mathrm{O,U}}$ (K)\\
        \hline
        0,8  & 7,13 & -8,97 \\
        1,6  & 1,11 FEHLER & -15,09 \\
        2,4  & -2,45 &  -18,65\\
        3,2  & -3,79 & -19,99 \\
        4,0  & -3,34 & -19,54 \\
        
    \end{tabular}
    \caption{Die Temperatur oben am Petierelement und der Bertrag der Differenz Unten-Oben für die verschiedenen Stromstärken}
    \label{tb:to}
\end{table}


\begin{figure}[H]
    \centering
    \includegraphics[width=\linewidth, keepaspectratio]{Bilder/deltatgegenI.png}
    \label{bi:DeltatgegenI}
    \caption{Der Betrag der Temperaturdifferenzen aus Tabelle \ref{tb:to} der jeweiligen Stromstärken. Mit einer Ausgleichsgerade. Die Ausgleichsgerade wurde mittels Numpy.polyfit erstellt. Es wurde ein Polynom 2ten Grades verwendet. } 
\end{figure}

Das Maximum der Kurve befindet sich bei $I = 3.34$ A. wo  ein Temperaturunterschied von $20.20$ °C herscht. Diese Werte haben aber keinen großen Erkenntnissgewinn da sie ganz Konkret von dem Aufbau des Petier-Elements abhäging sind, z.B der Isolierung, dem Ohmschen Wiederstand etc. Die Vermutug das es ein Solches Maximum für die Kühlleistung gibt hat sich aber bestätigt. Ausserdem ist ein Exponentieller Zusammenhang zwischen $I$ und $\Delta T$ festgestellt. MEHR

Nun Soll der Peltier-Koeffizient $\Pi$ mit Gleichung \ref{effektive_Kühlung} berechnet werden. Formt man diese zu 

\begin{equation}
\frac{\Delta T_{\mathrm{O,U}}}{I} = \frac{1}{2c} \cdot U - \frac{\Pi}{c}
\label{gl:toui}
\end{equation}

um, so kann man c mit Hilfe der Steigung der Ausgleichsgerade in Abbildung \ref{} errechen. 

\begin{figure}[H]
    \centering
    \includegraphics[width=\linewidth, keepaspectratio]{Bilder/tiu.png}
    
    \caption{Die Temperaturdifferenzen aus Tabelle \ref{tb:to} geteilt durch die jeweiligen Stromstärken aufgetragen gegen die Spannungen.  Mit Ausgleichs- und Grenzgeraden. Die Ausgleichsgerade wurde mittels Numpy.polyfit erstellt. Es wurde ein Polynom 1ten Grades verwendet. ie Grenzgeraden Mittels einer Funktion welche die Grenzeraden so legt, dass $\frac{2}{3}$ aller Punkte zwischen den beiden Grenzgeraden liegen. Alle drei Geraden laufen durch den Schwerpunkt.} 
    \label{bi:tiu}
\end{figure}

Die Steigungen $m$ und die y-Achsenabschnitte $t$ von den Graphen in Abbildung \ref{bi:tiu} sind folgender Tabelle \ref{tb:steig2} zu sehen.

\begin{table}[H]
    \centering
    \begin{tabular}{c  c c}
        \hline
        Gerade &  Steigung ($m \frac{\mathrm{K}}{\mathrm{W}}$) & y-Achsenabschnitt $t $ $(\frac{\mathrm{K}}{\mathrm{A}})$\\
        \hline
        Ausgleichsgerade   & 0,597 & -12,782   \\
        Grenzgerade 1   & 0,545 & -12,350   \\
        Grenzgerade 2 & 0,655 & -13,256   \\
    
    \end{tabular}
    \caption{Die Steigungen der Ausgleichs- und Grenzgeraden}
    \label{tb:steig2}
\end{table}


Aus Gleichung \ref{gl:toui} folgt
\begin{equation}
m = \frac{1}{2c} \Leftrightarrow c = \frac{1}{2m} = 0,83 \frac{\mathrm{W}}{\mathrm{K}}
\end{equation}

für die Proportionalitätskonstante $c$ und dann für den Peltier-Koeffizient
\begin{equation}
t = - \frac{\Pi}{c} \Leftrightarrow \Pi = -t \cdot  c = 10,615 V
\end{equation}

Dabei Werden die Fehler der Steigung und des y-Achsenabschnitts wie bei dem Fehler des Seebeck-Koeffizient Gleichung (\ref{seefehler}) bestimmt. Mit Werten aus Tablle \ref{tb:steig2} ergibt sich

\begin{equation}
\Delta m = 0,055~,~\Delta t = 0,453
\end{equation}

Womit sich dann für die abgeleitete Größe c ein Fehler von 

\begin{equation}
\Delta c = \pm \left( \left| \frac{\partial c}{\partial m} \right| \Delta m \right)
           = \pm \frac{\Delta m}{2 m^{2}}
           \approx \pm 0{,}040\,\frac{\mathrm{W}}{\mathrm{K}} \, .
\end{equation}
ergibt. Der Fehler des Peltier-Koeffizienten ist durch
\begin{equation}
\Delta \Pi
= \pm \left( 
\left| \frac{\partial \Pi}{\partial c} \right| \Delta c
+ 
\left| \frac{\partial \Pi}{\partial t} \right| \Delta t
\right)
= \pm (t \cdot \Delta c + c \cdot \Delta t)
\approx \pm 0{,}781\,\mathrm{V}.
\end{equation}

gegeben. Während der relativ Fehler von c, $\frac{\Delta c}{c} = 4,9 \%$ noch relativ gering ist, Ist der relativ Fehler von $\Pi$, $\frac{\Delta \Pi}{\Pi} = 7,4$ schon signifikant. Dies kommt einerseits davon, dass mit dem Größtfehler gerrechnet wurde, was bei einer Größe die von meherern fehlerbehafteten Größen abhängt automatisch zu einem großen Fehler führt. Nun soll die joulsche-Wärmeleistung und die joulsche-Wärmeleistung seperat voneinander betrachtet werden. Trägt man beide Leistungen gegen die Stromstärke $I$ auf so erhält man Abbildung \ref{bi:Leistungen}.

\begin{figure}[H]
    \centering
    \includegraphics[width=\linewidth, keepaspectratio]{Bilder/Leistungen.png}
    \label{bi:Leistungen}
    \caption{Die Peltier-Wärmeleistung in Rot und die joulsche-Wärmeleistung in Schwarz aufgetragen gegen die Stromstärke.  Die Ausgleichsgerade der joulsche-Wärmeleistung wurde mittels Numpy.polyfit erstellt. Es wurde ein Polynom 2ten Grades verwendet. } 
\end{figure}

Wie Vermutet ist die joulsche-Wärmeleistung exponentiell von der Stromstärke abhängig. Das lässt sich erklären in dem man das Peltier-Element als ohmeschen Wiederstand annimmt und somit für die Wärmeleistung 

\begin{equation}
P = R \cdot I^2
\end{equation}

folgt. Die Peltier-Wärmeleistung verläuft linear, ... Des Weitern kann die Vermutung die aus Abbildung \ref{bi:Utversc} aufgestellt wurde, dass bei der Stromstärke von $I = 4$A die joulsche-Wärmeleistung stärker als die Peltier-Wärmeleistung steigt hat sich bestätigt. Während die Steigung der Peltier-Wärmeleistung konstant $\Pi = 10,615 \frac{\mathrm{W}}{\mathrm{A}} $ beträgt, hat die joulsche-Wärmeleistung am Punkt $I = 4$A eine Steigung von $27.74 \frac{\mathrm{W}}{\mathrm{A}}$ 

\subsection{Seebeck-Koeffizient des Peltier Elements}

Nun soll der Seebeck-Koeffizient de Peltier Elements bestimmt werde. Hierführ muss zuerst der Temperatur Unterschied zwischen der Ober- und Unterseite des Peltier errechnet werden. Dies kann mit Gleichung (\ref{gl:uth}) mit den Gemmessenen Thermospannungen $U_{\mathrm{th}}$ und dem Seebeck-Koeffizient welcher in Kapitel 4.1 Berechnet wurde, getan werden. In Tabelle sind die daraus folgenden Temperaturdifferenzen aufgetragen, wobei die Temperatur an der unterseite des Peltier-Elements in einem Bereich von 16,7°C und 15,0°C war.

\begin{table}[H]
    \centering
    \begin{tabular}{c | c | c}
        \hline
        $U_{\mathrm{th}}$ (V) & $T_{\mathrm{O}}$ (°C) & $\Delta T_{\mathrm{O,U}}$ (K) \\
        \hline
        -0.17 & -3.79 & 20.49 \\
        -0.17 & -3.79 & 20.09 \\
        -0.12 & -2.67 & 17.67 \\
        -0.08 & -1.78 & 16.78 \\
         0.00 &  0.00 & 15.10 \\
         0.05 &  1.11 & 14.19 \\
         0.10 &  2.23 & 13.17 \\
         0.15 &  3.34 & 12.06 \\
         0.20 &  4.45 & 11.05 \\
         0.25 &  5.57 & 10.03 \\
         0.30 &  6.68 &  8.92 \\
         0.35 &  7.80 &  7.90 \\
         0.40 &  8.91 &  6.79 \\
         0.45 & 10.02 &  5.78 \\
         0.50 & 11.14 &  4.66 \\
         0.55 & 12.25 &  3.55 \\
         0.59 & 13.14 &  2.66 \\
         0.61 & 13.59 &  2.31 \\
         0.65 & 14.48 &  1.42 \\
         0.70 & 15.59 &  0.31 \\
         0.75 & 16.70 & -0.80 \\
         0.78 & 17.37 & -1.37 \\
        \hline
    \end{tabular}
    \caption{Die Temperatur oben am Peltier-Element und der Betrag der Differenz Unten-Oben für die verschiedenen Stromstärken}
    \label{tb:to3}
\end{table}
 

Nun kann man den Peltier-Koeffiezient wie in Kapitel 4.1 bestimmen. Dafür wurde die Thermospannung $U_{\mathrm{Pe}}$ gegen die Temperaturdifferenz $T_{\mathrm{O,U}} - T_{\mathrm{ref}} = T_{\mathrm{O,U}}$, da $ T_{\mathrm{ref}} = 0°C $ist, aufgetragen. Zudem wurden die Ersten zwei Wertepaare nicht mit einbezogen, weil $U_{\mathrm{Pe}}$ bei diesen werten $> 5V$ ist und Somit nicht von dem Seebeck Effekt, sondern noch von der Angelegten Spannung stammt.  Das Ergebniss ist in Abbildung (\ref{letzepls}) zu sehen.

\begin{figure}[H]
    \centering
    \includegraphics[width=\linewidth, keepaspectratio]{Bilder/Figure_5.png}
    \caption{Die Peltier-Wärmeleistung in Rot und die joulsche-Wärmeleistung in Schwarz aufgetragen gegen die Stromstärke.  Die Ausgleichsgerade der joulsche-Wärmeleistung wurde mittels Numpy.polyfit erstellt. Es wurde ein Polynom 2ten Grades verwendet. } 
    \label{letzepls}
\end{figure}


Die Steigungen sind in Tabelle \ref{tb:steig3} zu sehen.

\begin{table}[H]
    \centering
    \begin{tabular}{c c}
        \hline
        Gerade &  Steigung ($\frac{\mathrm{mV}}{\mathrm{K}}$)\\
        \hline
        Ausgleichsgerade   & 0,0549   \\
        Grenzgerade 1   & 0.0424   \\
        Grenzgerade 2 & 0.0712   \\
    
    \end{tabular}
    \caption{Die Steigungen der Ausgleichs- und Grenzgeraden}
    \label{tb:steig3}
\end{table}

Damit lässt sich Der Seebeck-Koeffizent nach Gleichung \ref{gl:uth}bestimmen. Der Fehler ist durch \ref{} gegeben. 
\begin{equation}
\alpha_{\mathrm{P}} = (5,49 \pm 1,44) \cdot 10^{-2}\frac{\mathrm{V}}{\mathrm{K}}
\end{equation}

FEHLER

Dieses Egebniss kann nun mit dem Seebeck-Koeffizent verglichen werden, der sich ergibt wenn man Gleichung \ref{} verwendet. Für die Temperatur wurde der Mittelwert des betrachteten Temperaturbereichs verwendet:
\begin{equation}
\alpha_{P,2} = \frac{\Pi}{T} = 3,73 \cdot 10^{-2} \frac{\mathrm{V}}{\mathrm{K}}
\end{equation} 

Wobei der Fehler durch
\begin{equation}
\Delta\alpha_{\text{P},2} = \pm \left( \left| \frac{\partial\alpha_{\text{P},2}}{\partial\Pi} \right| \Delta\Pi \right) = \pm \frac{\Delta\Pi}{T} \approx \pm 0{,}27 \cdot 10^{-2} \frac{\text{V}}{\text{K}}
\end{equation}
gegeben ist. Somit überschneiden sich diese zwei Ergebnisse auch mit dem Maximalfehler um $0,05 \cdot 10^{-2} \frac{\text{V}}{\text{K}}$ nicht. Zudem ist der Fehler des Seebeck-Koeffizent welcher mithilfe der Grafik erstellt wurde sehr größ, der relativ Fehler beträgt ca. 26\%. Der Verlauf der Punkte in Grafik \ref{tb:steig3} lässt auch eine Exponentielle Komponente vermuten. Möglich ist das wie in Kapitel 4.2 durch den fließenden Strom, der jetzt hier nicht gemessen wurde, joulsche Wärme entsteht, diese ist wie oben schon gezeigt quadratisch von der Stromstärke abhängig. Trotzdem befinden sich die zwei errechnetet Seebeck-Koeffizenten in der gleichen Größenordnung, der unterschied liegt bei einem Faktor von ca. 1,5, somit liegen beide Ergebnisse ingesamt auch in einer realistischen Größenordnung. Der Unterschied zum Seebeck-Koeffizent des Thermoelemts hingegen ist sehr groß der Faktor betragt hier ungefähr 1000, wobei der Seebeck-Koeffizent des Petier-Elements deutlich größer als der des Thermoelemts ist. Das beteutet das, dass Petier-Element deutlich effizienter Wärme in elektrische Spannung umzuwandeln kann. Wenn man die Gleichung 
\begin{equation}
\alpha_{\text{AB}} = \alpha_{\text{A}} - \alpha_{\text{B}}
\end{equation}

betrachtet heißt das zum einen das im Peltier-Element der unterschied der Seebeck-Koeffizenten der zwei Stogge tendenziell höher ist, zum anderen das der Stoff A tendenziell einen höhern Seebeck-Koeffizent als sein gegenstück vom Thermo-Element hatt. Der große Untscheid der Seebeck-Koeffizent beim Peltier-Element ergibt sinn, da es mit einer differenz der Fermienergie der Materialen arbeitet, welche nach Gleichung \ref{material_abhängig_seebeck}den Seebeck-Koeffizenten mitbestimmen. 