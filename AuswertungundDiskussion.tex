\section{Auswertung}

\subsection{Seebeck-Koeffizient des Thermoelements}

Um den Seebeck-Koeffizient des Thermoelements zu bestimmen kann eine Gleichung für zwei Materialien A und B und bei einem Temperaturgefälle von $T_2 - T_1$  aus Gleichung (\ref{}) hergeleitet werden. 
\begin{equation}
  U_{th} = \int_{T_1}^{T_2} \alpha_A~ \text{d}T - \int_{T_1}^{T_2} \alpha_B~ \text{d}T = \int_{T_1}^{T_2} = \int_{T_1}^{T_2} \alpha_{AB}~ \text{d}T
\end{equation}
Mit $\alpha_{AB} = \alpha_A \alpha_B$ vereinfacht sich das zu

\begin{equation}
U_{th} = \alpha_{AB} \cdot (T_2 - T_1)
\end{equation}

Trägt man die Temperaturdifferenz, welche in diesem Fall einfach die gemessene Temperatur gegen die gemessene Spannung auf so erhält man folgendes Diagramm.