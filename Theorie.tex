\section{Theoretische Grundlagen}

\subsection{Seebeck-Effekt}
Der Seebeck-Effekt ist einer der drei thermoelektrischen Effekte. Er tritt innerhalb eines homogenen Leiters auf, wenn ein Temperaturgefälle $\Delta T$ existiert. Dieses Temperaturgefälle löst innerhalb des Leiters einen Diffusionsstrom aus. Durch diesen Diffusionsstrom baut sich zusätzlich ein elektrisches Feld auf, das die Ladungstrennung begrenzt. Bei zeitlich konstantem $\Delta T$ stellen sich schließlich stationäre Verhältnisse ein. Dadurch bildet sich eine Thermospannung $U_\text{th}$.

Diese Thermospannung kann jedoch nicht direkt gemessen werden; hierfür wird ein zweiter Leiter benötigt. Das Verhältnis der Thermospannung $U_\text{th}$ zur Temperaturdifferenz $\Delta T$ wird als Seebeck-Koeffizient bezeichnet und ist durch
\begin{equation}
    \centering
    \alpha = \frac{dU_{\text{th}}}{dT}
    \label{Material_unabhängiger_Seebeck_1}
\end{equation}
gegeben. Wenn die Elektronen als homogenes Gas innerhalb des Leiters angenommen werden, kann
\begin{equation}
    \centering
    \alpha = \frac{3}{2} \frac{k_\text{B}}{e}
    \label{Material_unabhängiger_Seebeck_2}
\end{equation}
verwendet werden. Diese Gleichung suggeriert jedoch, dass $\alpha$ materialunabhängig sei. Experimentell wurde diese Materialunabhängigkeit widerlegt. Der Grund dafür liegt darin, dass sich die freien Elektronen nicht exakt wie ein ideales Gas verhalten.

Um diese Abweichung zu korrigieren, wird ein Korrekturterm eingeführt. Dabei wird berücksichtigt, dass die freien Elektronen des Leiters bereits kinetische Energie besitzen. Zur Bestimmung dieser kinetischen Energie wird das Bändermodell betrachtet. Dieses besagt, dass bei $T = 0\,\text{K}$ nur die energetisch niedrigsten Zustände besetzt sind; der höchste besetzte Zustand wird als Fermi-Energie bezeichnet. Bei Temperaturen $T >> 0$ verschwimmt die scharfe Grenze zwischen besetzten und unbesetzten Zuständen, da sich die Elektronen nun bewegen können. Die Breite des Bereichs, in dem besetzte und unbesetzte Zustände gleichzeitig auftreten, beträgt ungefähr $k_\text{B}T$.

Zusätzlich ist die Verteilung der Zustände nicht gleichmäßig. Die Zahl möglicher Energiezustände steigt gemäß
\begin{equation}
    \centering
    \rho(E) \propto \sqrt{E}
    \label{Dichte}
\end{equation}
mit der Energie. Mithilfe dieser Relation kann ein Flächenverhältnis von $\frac{3}{2}$ hergeleitet werden (grafische Darstellung in Abbildung …). Daraus ergibt sich schließlich der materialabhängige Seebeck-Koeffizient:
\begin{equation}
    \centering
    \alpha = \frac{9}{2} \frac{k_\text{B}}{e} \frac{T}{T_\text{F}}
    \label{material_abhängig_seebeck}
\end{equation}

Um nun die Thermospannung zweier Leitermaterialien zu berechnen, wird Gleichung \ref{Material_unabhängiger_Seebeck_1} nach $U_\text{th}$ umgestellt und integriert. Daraus folgt:
\begin{equation}
    \centering
    U_{\text{th}} = \alpha_{\text{AB}} \left(T_2 - T_1\right)
    \label{Thermospannung}
\end{equation}
als Formel der materialabhängigen Thermospannung.

\subsection{Peltier-Effekt}

\begin{figure}[H]
    \centering
    \includegraphics[scale = 0.5]{Bilder/Peltier-Effet.png}
    \caption{Darstellung des Peltier-Effekts anhand des Bändermodells \cite{W9}}
    \label{PeltierBild}
\end{figure}

Der Peltier-Effekt ist der zweite der drei thermoelektrischen Effekte. Bei ihm wird durch einen fließenden Strom am Übergang zweier Leiter Wärmeenergie aufgenommen oder abgegeben. Ursache dafür ist die unterschiedliche Fermi-Energie der beiden Materialien. Da der Elektronenfluss in beiden Leitern gleich sein muss, jedoch die Leitungsbänder verschiedene Energieniveaus besitzen, müssen Elektronen beim Übergang entweder potenzielle Energie aufnehmen oder abgeben, um weiterfließen zu können. In Abbildung \ref{PeltierBild} wird dieses Phänomen grafisch veranschaulicht.

Für zwei Leiter ist die transportierte Wärmeleistung gegeben durch
\begin{equation}
    \centering
    \dot{Q}_{\text{AB}} = \frac{E_\text{A} - E_\text{B}}{e} \cdot I = \Pi_{\text{AB}} \cdot I
    \label{Peltier_koeffiezient_2}
\end{equation}
Der Peltier-Koeffizient eines einzelnen Leiters lautet
\begin{equation}
    \centering
    \dot{Q}_{\text{A}} = \frac{E_\text{A}}{e} \cdot I = \Pi_{\text{A}} \cdot I
    \label{Peltier_koeffiezient_1}
\end{equation}

Der Peltier-Koeffizient lässt sich umformen zu
\begin{equation}
    \centering
    \Pi_{\text{A}} = \frac{E_\text{A}}{e} = \frac{3}{2} \frac{k_\text{B}}{e} T = \alpha T
    \label{Seebeck_x_Peltier}
\end{equation}
wodurch der Zusammenhang zwischen Seebeck- und Peltier-Koeffizient sichtbar wird.

Da dieser Prozess, wie alle realen Prozesse, nicht vollständig effizient abläuft, muss die zusätzlich entstehende ohmsche Wärme berücksichtigt werden. Die ohmsche Leistung $P = UI$ verteilt sich gleichmäßig auf die Ober- und Unterseite des Elements. Für die effektive Kühlleistung der kalten Seite ergibt sich daher:
\begin{equation}
    \centering
    c \cdot \Delta T = \frac{1}{2} U I - \Pi_{\text{AB}} I
    \label{effektive_Kühlung}
\end{equation}

\subsection{Thomson-Effekt}

Der Thomson-Effekt ist der dritte thermoelektrische Effekt. Im Gegensatz zu den beiden anderen Effekten tritt er direkt entlang eines stromdurchflossenen Leiters auf, wenn ein Temperaturgefälle anliegt. Dabei wird entlang des Leiters eine Wärmeleistung $\dot{Q}$ frei. Die Ursache liegt darin, dass die Fermi-Energie temperaturabhängig ist und sich somit entlang des Temperaturgefälles ändert.



