\section{Theoretische Grundlagen}

\subsection{Seebeck-Effekt}
Der Seebeck-Effeckt ist einer von den drei Thermoelektrischen-Effekten, 
er tritt innerhalb eines homogenen Leiters auf wenn ein 
Temperaturgefälle $\Delta T$ existiert. Dieses Termperatur Gefälle 
löst innerhalb des Leiters einen Diffusionsstrom aus. Durch diesen 
Diffusionsstrom baut sich zusätzlich nun ein Elektrisches Feld auf 
welches die Ladunstrennung begrenzt. Bei zeitlich gleichbleibendem 
$\Delta T$ stellen sich nun stationäre Verhältnisse ein. 
Aufgrunddessen bildet sich eine Thermospannung $U_{thermo}$. 
Diese kann aber nicht direkt gemessen werden, um diese zu messen 
wird ein zweiter Leiter benötigt. Das Verhältniss der Thermospannung 
$U_{thermo}$ zur Temperaturdifferenz $\Delta$T wird Seebeck Koeffizient 
genannt und ist durch
\begin{equation}
    \centering
\alpha = \frac{dU_{thermo}}{dT}
\label{Material_unabhängiger_Seebeck_1}
\end{equation}
gegeben. Wenn die Elektronen als homogenes Gas innerhalb des 
Leiters angenommen werden kann 
\begin{equation}
    \centering
\alpha = \frac{3}{2} \cdot k_B \cdot T
\label{Material_unabhängiger_Seebeck_2}
\end{equation}
als Formel verwendet werden. Wobei Formel \ref{Material_unabhängiger_Seebeck_2} 
Materialunabhängig ist und somit nur bedingt Korrekt ist.


\subsection{Peltier-Effekt}

\subsection{Thomson-Effekt}


